%\documentclass{Physics_H_Notes}
%\begin{document}
\chapter[热力学]{\itr{Thermodynamics}{热力学}}
    \section[导论]{\itr{Introduction}{导论}}
        假设一个有$N$个粒子的体系且经典力学分析依然成立,那么对第$i$个粒子,其受到其余$N-1$个粒子的作用,其动力学方程应有如下形式:
        \begin{equation}
            \sum_{i\neq j}^N \vec{F_{ij}} = \mu_j \vec{a_j}
            \nonumber
        \end{equation}    
        其中共有$N$个变量,故对整个系统将会列出共计$N^2$个变量的方程,当$N$足够大时求解这样的方程显然是极其繁琐且不现实的,
        于是就引出了\itr{Statistical Physics}{统计物理学}与\itr{Themiodynamics}{热力学}\linebreak 这两个研究超多粒子体系的分支。

        \itr{Statistical Physics}{统计物理学}从微观角度出发,利用统计学方法对宏观物体的热运动性质及其规律做出解释,
        本章中将会介绍其中的分子动理论。

        \itr{Thermodynamics}{热力学}从宏观角度研究物质的热运动性质及其规律,并给出少数关键宏观量的联系。
        本章将重点介绍这些关键宏观量以及确立它们之间联系的热力学定律。
    \section[理想气体]{\itr{Ideal Gas}{理想气体}}
        \itr{Ideal Gas}{理想气体}是一种理想模型,它满足以下条件:
        \\
        \begin{Itemize}
            \item 包含大量的分子,且分子自身占据的体积对分子间距而言可以忽略。
            \item 每个分子均可视为质点,服从牛顿运动定律;但作为一个整体,它们随机运动且速度分布与时间无关。
            \item 分子间以及分子与容器壁的碰撞都是弹性碰撞,且除分子的相互碰撞外忽略分子之间的作用力。
            \item 所有气体分子是相同的。
        \end{Itemize}
        \newpage
        理想气体适合\itr{Ideal Gas Law}{理想气体定律}\quad$pV=nRT=Nk_{B}T$,其中涉及的物理量与常数:
        \\
        \begin{Itemize}
            \item \itr{Amount of Substance}{物质的量}用于衡量分子数量,用$n$表示,国际单位为摩尔($\rm{mol}$)。
            \item \itr{Avogadro Constant}{阿伏伽德罗常数},记为$N_{A}$,约为$6.022\times 10^{23}\rm{mol}^{-1}$。
            表示$1mol$物质中含有的粒子数,即若用$N$代表分子个数,其关系为$N=nN_{A}$。
            \item \itr{Temperature}{热力学温度}表示气体热运动的剧烈程度,用$T$表示,国际单位为开尔文($\rm{K}$)。
            其与摄氏温度的转化为:$T = t + 273.15$,$t$为摄氏温度。
            \item \itr{Boltzmann Constant}{玻尔兹曼常数},记为$k_{B}$,约为$1.381\times 10^{-23}\rm{J}\cdot K^{-1}$。
            反应微观状态数与熵的关系。
            \item \itr{Gas Constant}{气体常数},记为$R$,约为$8.314\rm{J}\cdot mol^{-1}\cdot K^{-1}$且$R=N_{A}k_{B}$。
        \end{Itemize}
    \section[分子动理论]{\itr{Kinetic Theory of Molecules}{分子动理论}}
        \subsection[压强的微观解释]{\itr{the Microscopic View of Pressure}{压强的微观解释}}
            从微观角度来说,压强的本质是气体分子对容器壁不断撞击的统计平均结果。根据理想气体的假设,并认为速度向各个方向的概率均等,
            可以得到压强与分子数密度以及速率的关系如下:
            \begin{equation}
                p = \rho \mu \overline{v_{x}^{2}}=\frac{1}{3}\rho \mu \overline{v^{2}}
                \nonumber
            \end{equation}
            其中$\rho = \frac{N}{V}$表示单位体积的分子数密度,$\mu$表示单个分子的质量,$\overline{v^{2}}$为速度平方的平均值
            (注意与速度平均值的平方$\overline{v}^2$区分)。

            利用这一结论,如果我们定义分子平均动能\ $\overline{\epsilon_{k}} = \frac{1}{2} \mu \overline{v^{2}}$,
            结合理想气体方程,经过简单推导,我们即可得到其表达式为:
            \begin{equation}
                \overline{\epsilon_{k}} = \frac{3}{2}k_{B}T
                \nonumber
            \end{equation}
            它表明分子平均动能只与温度有关。
        \subsection[能量均分原理]{\itr{Equipartition Theorem}{能量均分原理}}
            首先介绍\itr{Degrees of Freedom}{自由度}的概念,它表示在坐标系中确定分子运动状态所需要的独立坐标数。总体来看,
            对于每个在三维坐标系中的原子,不考虑其内部的电子与原子核,我们需要3个独立坐标来确定其位置,如笛卡尔坐标系中的$x$、$y$、$z$,
            球坐标系中的$r$,$\theta$,$\phi$。故对于一个由$k$个原子组成的分子,其自由度总量为$3k$。而自由度又分为三个部分:
            \begin{Itemize}
                \item \itr{Translational Degree of Freedom}{平动自由度}表示确定分子质心在坐标系中的位置所需要的独立坐标数,
                对所有分子均为3,并且单原子分子只有平动自由度。
                \item \itr{Rotational Degrees of Freedom}{转动自由度}表示确定分子在坐标系中的空间取向所需要的独立坐标数。
                对于线性分子(如$\rm{CO_{2}}$),由于其绕通过分子的轴的旋转不改变分子构型,故旋转自由度为2,其余分子为3。
                \item \itr{Vibrational Degrees of Freedom}{振动自由度}表示确定分子内原子之间的相对运动状态所需要的独立坐标数。
                通常利用总自由度-平动自由度-转动自由度的方式计算,对线性分子为$3k-5$,对非线性分子为$3k-6$。
                %\footnote{如要直接分析,则振动自由数等于简正模式的数量,即彼此正交且不改变质心位置与分子取向的基本振动的数量}
            \end{Itemize}
            \ \newline 
            有了自由度的概念,根据大量的实验事实总结有如下关系成立:
            \begin{equation}
                U = \frac{i}{2}Nk_{B}T
                \nonumber
            \end{equation}
            其中$U$为\itr{Internal Energy}{内能},表示气体的所有分子动能、分子势能、分子内部能量等之和,
            但由于理想气体不计分子间作用与分子内部结构,故其理想气体内能等于理想气体的总动能;$i$为平动自由度与转动自由度之和
            \footnote{不计振动自由度、电子自由度与原子核自由度,这是因为这些自由度对应的能级能隙的数量级与$k_{B}T$相当(振动能隙)
            或远大于它(电子能隙与原子核能隙),均分定理所需的假设“能级组成平滑连续能谱”不再成立,其贡献变为0。
            事实上当温度降低到一定值时,这一定理也会因为同样的原因失效。};
            而$\frac{1}{2}k_{B}T$正是上一部分中我们计算得到的分子动能平分到某个方向上的结果,
            这一结论也由此解释为内能平均分配到每一个自由度上,得名\itr{Equipartition Theorem}{能量均分原理}。
        \subsection[麦克斯韦分布]{\itr{Maxwell Distribution}{麦克斯韦分布}}
            速率分布函数是这样一个函数,它的自变量为速度,因变量为速度落在$[v,v+\dif{v}]$区间内的粒子占总粒子数的比值,
            定义式如下:
            \begin{equation}
                f(v) = \frac{\dif{N}}{N\dif{v}}
                \nonumber
            \end{equation}
            这样速度落在$[v_{1},v_{2}]$区间内的粒子占总粒子数的比值即为:
            \begin{equation}
                \int_{v_{1}}^{v_{2}} f(v)\dif{v}
                \nonumber
            \end{equation}
            根据定义,速率分布函数是归一化的,即其在$[0,+\infty)$上的积分为1。
                
            \itr{Maxwell Distribution}{麦克斯韦分布}给出了理想气体的速率分布规律,它给出的速率分布函数如下:
            \begin{equation}
                f(v)dv=\left(\frac \mu{2\pi k_{B}T}\right)^{\frac{3}{2}}e^{-\frac{\mu v^{2}}{2k_{_B}T}}4\pi v^2\dif{v}
                \nonumber
            \end{equation}
            
            可以将麦克斯韦分布分为三个部分,第一部分为$e$指数前的部分,事实上是归一化系数;第二部分为$e$指数部分,
            揭示了速度分布与动能以及温度的关系;第三部分$4\pi v^{2}\dif{v}$则为在速度空间中的球壳体积元。

            利用速率分布函数,我们可以计算如下的特征速度:
            \begin{Itemize}
                \item \itr{Most Probable Speed}{最概然速率}表示粒子数最多的区间微元对应的速度,定义式为:
                \begin{equation}
                    v_{p} \quad where \quad \frac{\dif N({v_{p}})}{\dif{v}} = 0
                    \nonumber
                \end{equation}
                对麦克斯韦分布,其值等于$\sqrt{\frac{2k_{B}T}{\mu}}$或$\sqrt{\frac{2RT}{M}}$,$M$为摩尔质量。
                \item \itr{Average Speed}{平均速率}表示所有粒子速率的平均值,用于计算分子自由程等,定义式为:
                \begin{equation}
                    \overline{v} = \int_{0}^{+\infty} vf(v)dv
                    \nonumber
                \end{equation}
                对麦克斯韦分布,其值等于$\sqrt{\frac{8k_{B}T}{\pi \mu}}$或$\sqrt{\frac{8RT}{\pi M}}$。
                \item \itr{Root Mean Speed}{方均根速率}表示所有粒子速度平方的平均值的平方根,用于计算分子动能等,定义式为:
                \begin{equation}
                    v_{rms} = \sqrt{\int_{0}^{+\infty} v^{2}f(v)dv}
                    \nonumber
                \end{equation}
                对麦克斯韦分布,其值等于$\sqrt{\frac{3k_{B}T}{\mu}}$或$\sqrt{\frac{3RT}{M}}$。
            \end{Itemize}
        \subsection[玻尔兹曼分布]{\itr{Boltzmann Distribution}{玻尔兹曼分布}}
            我们再重新观察一下麦克斯韦分布的$e$指数位置,发现可以提出$\frac{1}{2}\mu v^{2}$,这正是分子动能,
            这启示我们粒子的分布的核心可能是它的能量。那么在一般的势场中,随着粒子所在位置的不同,自然其能量不同,
            所以对应的分布应该与空间坐标有关。

            事实也确实如此,在同时考虑速度与位置的情况下,我们有如下的关系:
            \begin{equation}
                \dif{N}=\rho_{0}\left(\frac \mu{2\pi k_{B}T}\right)^{\frac{3}{2}}e^{-\frac{\epsilon}{k_{_B}T}}
                \dif{x}\dif{y}\dif{z}\dif{v_{x}}\dif{v_{y}}\dif{v_{z}}
                \nonumber
            \end{equation}
            其中$\rho_{0}$为势能零点处的分子数密度。
            仔细观察可见,若将内能拆成势能与动能的和,再分别与对应的变量组合,就有:
            \begin{equation}
                \dif{N} =\rho_{0}e^{-\frac{\epsilon_{p}}{k_{_B}T}}\dif{x}\dif{y}\dif{z}\times\left(\frac \mu{2\pi k_{B}T}\right)^{\frac{3}{2}}e^{-\frac{\epsilon_{k}}{k_{_B}T}}
                \dif{v_{x}}\dif{v_{y}}\dif{v_{z}}
                \nonumber
            \end{equation}
            由于$\dif{v_{x}}\dif{v_{y}}\dif{v_{z}}=4\pi v^{2}\dif{v}$事实上是速度在三维微分的不同表达形式,
            此时后一项即是麦克斯韦分布,积分为1,故对两边进行关于速度的积分,就有:
            \begin{equation}
                \dif{N} =\rho_{0}e^{-\frac{\epsilon_{p}}{k_{_B}T}}\dif{x}\dif{y}\dif{z} =\rho_{0}e^{-\frac{\epsilon_{p}}{k_{_B}T}}\dif{V}
                \nonumber
            \end{equation}
            接下来只要注意到$\rho = \frac{\dif{N}}{\dif{V}}$,就自然得到了一个简洁而重要的结论:
            \begin{equation}
                \rho = \rho_{0}e^{-\frac{\epsilon_{p}}{k_{_B}T}}
                \nonumber
            \end{equation}
            这一分布称为\itr{Boltzmann Distribution}{玻尔兹曼分布},
            我们将处于势能零点的粒子称为基态粒子,那么这个式子反映了能量高于$\epsilon_{p}$的粒子相对基态粒子的数量。
            
            利用它,我们可以轻松推出流体力学中压强随高度(假设温度不变)的分布,只需注意到压强与分子数成正比,
            势能在此处为重力势能即可,有:
            \begin{equation}
                p = p_{0}e^{-\frac{\mu gh}{k_{_B}T}} = p_{0}e^{-\frac{mgh}{RT}}
                \nonumber    
            \end{equation}
        \subsection[平均自由程]{\itr{Mean Free Path}{平均自由程}}
            \itr{Mean Free Path}{平均自由程}是指分子在相邻两次碰撞之间走过的平均路程,它等于分子在一段时间内通过的路程与平均碰撞次数的比值。
            如将分子视为直径为$d$的球体且分子做平均速度为$v$的折线运动,可以得到平均碰撞次数$\overline{Z}$为:
            \begin{equation}
                \overline{Z} = \sqrt{2}\pi \rho d^{2} v
                \nonumber
            \end{equation}

            那么对应的平均自由程$l$为:
            \begin{equation}
                l =  \frac{1}{\sqrt{2}\pi \rho d^{2}} = \frac{k_{_B}T}{\sqrt{2}\pi d^{2}p}
                \nonumber
            \end{equation}
        \subsection[范德华状态方程]{\itr{Van Der Waals Equation of State}{范德华状态方程}}
            对真实气体,范德华提出了一个基于理想气体的修正模型:
            \begin{equation}
                (p+\frac{a}{V_{m}^{2}})(V_{m}-b)=RT
                \nonumber
            \end{equation}
            其中$V_m$表示气体的摩尔体积,压强的附加项考虑分子间的引力,体积的附加项考虑的分子自身占据的体积。
    \section[热力学第零定律]{\itr{Zeroth Law of Thermodynamics}{热力学第零定律}}
        \subsection[引言]{Introduction}
            从这节从此开始均为热力学部分,在开始之前,想阐述(吐槽)一下热力学学习中可能遇到的一些问题,复习/补天选手可以跳过。
           
            关于各个课程热力学量符号各成一体的问题,以功为例。在高中课程与普通物理中均用$W$表示物体对外界做功,
            但高中时规定对内做功为正,普通物理规定对外做功为正;大学物理(甲)则用$A$表示外界对物体做功,
            且外界对物体做正功时为正;而物理化学则用$w$表示物体对外界做功并强调它不是一个状态量,规定对内做功为正。
            所以说在学习这一块的时候,请各位务必注意符号问题,很有可能课程或参考书的符号及其定义与本笔记有较大出入。

            关于热力学函数的$\dif{Q}$这类对于过程函数的微分,最好从物理上理解为一个微小可逆过程的放热,
            因为如果把$Q$作为一个数学概念上的多元函数,那么它并不能全微分,自然不存在$\dif{Q}$这一说法。
            而对于状态函数的微分如$\dif{U}$,其可以视为数学上的全微分。

            然后是贯穿本章的一个问题,这部分选取的统计力学与热力学由于是比较原始的部分,
            学完总会给人一种这章全是记忆性知识,到处给理论打补丁的感觉,不像力学与电磁学那样能自成体系,
            所以在证明的第二部分特意给出了这些定律到底是如何从底层给出的,供感兴趣者阅读。
        \subsection[基本概念]{\itr{Basic Concepts}{基本概念}}
            \begin{Itemize}
                \item \itr{System}{系统}是选定作为研究对象的那部分物质及其所占有的空间,系统是有\itr{Boundary}{边界}\linebreak 的,
                系统与环境的物质与能量交换均发生在边界上。
                \item \itr{Surroundings}{环境}是除了系统之外的部分,可以与系统交换物质与能量。
                \item \itr{Open System}{开放系统}与环境既有物质交换,又有能量交换。
                \item \itr{Closed System}{封闭系统}与环境只有能量交换,没有物质交换。
                \item \itr{Isolated System}{孤立系统}与环境既无物质交换,又无能量交换。
                \item \itr{Thermodynamic equilibrium}{热力学平衡态}是系统内的各性质与组分保持稳定的一种状态,
                它要求系统达到与环境的力平衡(合外力为0,刚性器壁除外),热平衡(无能量交换)
                以及系统的相平衡(各相变过程达到平衡)与化学平衡(各化学反应达到平衡)。之后讨论的状态均为热力学平衡态。
                \item \itr{State Function}{状态函数}是函数值只与状态有关的函数,其在两个态之间的变化量与路径无关。
                在数学上即其因变量在定义域内可表示为自变量的全微分。
                \item \itr{Process Function}{过程函数}与状态函数相对,在两个态之间的函数变化量与路径有关。
            \end{Itemize}
        \subsection[温度]{\itr{Temperature}{温度}}
            \itr{Temperature}{温度}是一个相对的概念,它的建立基于\itr{Zeroth Law of Thermodynamics}{热力学第零定律}
            \begin{law}[Zeroth Law of Thermodynamics]
                若两个热力学系统均与第三个系统处于热平衡状态,此两个系统也必互相处于热平衡。
            \end{law}
          
            它揭示了热平衡的可传递性,这样只需选定两个温度作为基准,就可以建立一个温度体系。

            热力学温度是以\itr{Absolute Zero}{绝对零度}为温度零点的,且温度线斜率与摄氏温度一致。
    \section[热力学第一定律]{\itr{First Law of Thermodynamics}{热力学第一定律}}
        \subsection[热与功]{\itr{Heat and Work}{热与功}}
            \begin{Itemize}
                \item \itr{Heat}{热}是由系统和周围环境之间的温差导致而在系统边界上传递的能量,用$Q$表示。热是过程函数。
                热的传递导致温度的传递,它们的关系如下:
                \begin{equation}
                    \frac{\dif{Q}}{\dif{t}} = \kappa_{t}A\frac{\dif{T}}{\dif{x}}
                \end{equation}
                这是\itr{Fourier's Law}{傅里叶定律},其中$\kappa_{t}$表示介质的热传导率,与介质种类有关;$A$为截面面积;
                $\frac{\dif{T}}{\dif{x}}$代表了温度梯度,即两端的温差大小。
                \item \itr{Heat Capacity at Constant Volume}{等容热容}是在体积一定的情况下上升单位温度所需要吸收的热量:
                \begin{equation}
                    C_{V} \overset{def}{=} \left(\frac{\partial{Q}}{\partial{T}}\right)_{V} = \left(\frac{\partial{U}}{\partial{T}}\right)_{V}
                    \nonumber
                \end{equation}
                对理想气体,设$i$为平动自由度与转动自由度之和:
                \begin{equation}
                    C_{V} = \frac{i}{2}R
                    \nonumber
                \end{equation}
                \item \itr{Heat Capacity at Constant Pressure}{等容热容}是在压强一定的情况下上升单位温度所需要吸收的热量:
                \begin{equation}
                    C_{p} \overset{def}{=} \left(\frac{\partial{Q}}{\partial{T}}\right)_{p} %= (\frac{\partial{H}}{\partial{T}})_{_p}
                    \nonumber
                \end{equation}
                对理想气体,设$i$为平动自由度与转动自由度之和:
                \begin{equation}
                    C_{p} = \frac{i+2}{2}R = C_{V}+R
                    \nonumber
                \end{equation}
                \item \itr{Work}{功}与力学时的定义一致,同样用$W$表示:
                \begin{equation}
                    W \overset{def}{=} \int F\dif{x} = \int pS\dif{x} = \int p\dif{V}
                    \nonumber
                \end{equation}
                并规定系统对外做功取正。功也是过程函数。                
            \end{Itemize}

            功与热是改变系统内能的两条也是唯二的途径,这就是\itr{First Law of Thermodynamics}{热力学第一定律}\linebreak
            \begin{law}[First Law of Thermodynamics]
                系统内能的增加量等于系统吸收的热和环境对系统所做的功的总和:
                \begin{equation}
                    \Delta U = Q - W
                    \nonumber
                \end{equation}
            \end{law}
            热力学第一定律也可以写成微分形式:
            \begin{equation}
                \dif{U} = \dif{Q} - p\dif{V}
                \nonumber
            \end{equation}
        \subsection[过程介绍]{\itr{Process Introduction}{过程介绍}\mgnote{为方便查阅把熵的计算也写在此处}}
        \begin{Itemize}
            \item \itr{Isochoric Process}{等容过程}是指系统体积不变的过程,此时一般容器为刚性容器。            
            \begin{equation}
                \begin{aligned}
                    &\frac{p}{T} = constant \\
                    &Q = \Delta U = nC_{V}\Delta T = \frac{i}{2}V\Delta p \\
                    &W = 0 \\
                    &\Delta U = nC_{V}\Delta T = \frac{i}{2}V\Delta p \\ 
                    &\Delta S = nC_{V}ln\frac{p_{2}}{p_{1}} =nC_{V}ln\frac{T_{2}}{T_{1}}
                \end{aligned}
                \nonumber
            \end{equation}
            \item \itr{Isobaric Process}{等压过程}是指系统压强不变的过程。
            \begin{equation}
                \begin{aligned}
                    &\frac{V}{T} = constant \\
                    &Q = nC_{p}\Delta T =\frac{i+2}{2}p\Delta V\\
                    &W = p\Delta{V} \\
                    &\Delta U = Q - W = nC_{V}\Delta T = \frac{i}{2}p\Delta V \\ 
                    &\Delta S = nC_{p}ln\frac{V_{2}}{V_{1}} =nC_{p}ln\frac{T_{2}}{T_{1}}
                \end{aligned}
                \nonumber
            \end{equation}
            \item \itr{Isothermal Process}{等温过程}是指系统温度不变的过程。
            \begin{equation}
                \begin{aligned}
                    &pV = constant \\
                    &Q = W = nRTln\frac{P_1}{P_2} = nRTln\frac{V_2}{V_1}\\
                    &W = nRTln\frac{P_1}{P_2} = nRTln\frac{V_2}{V_1} \\
                    &\Delta U = 0 \\ 
                    &\Delta S = nRln\frac{V_{2}}{V_{1}} =nRln\frac{P_{1}}{P_{2}}
                \end{aligned}
                \nonumber
            \end{equation}
            \item \itr{Adiabatic Process}{绝热过程}是指系统不与环境交换能量的过程。
            \begin{equation}
                \begin{aligned}
                    &pV^{\gamma} = constant \qquad \gamma = \frac{C_p}{C_V} = \frac{i+2}{i}\\
                    &Q = 0\\
                    &W = \frac{p_{1}V_{1}^{\gamma}}{1-\gamma}\left(\frac{1}{V_{2}^{\gamma-1}}-\frac{1}{V_1^{\gamma-1}}\right)\\
                    &\Delta U = -W =\frac{p_{1}V_{1}^{\gamma}}{\gamma - 1}\left(\frac{1}{V_{2}^{\gamma-1}}-\frac{1}{V_1^{\gamma-1}}\right)\\ 
                    &\Delta S = 0
                \end{aligned}
                \nonumber
            \end{equation}
        \end{Itemize}
    \section[热力学第二定律]{\itr{Second Law of Thermodynamics}{热力学第二定律}}
        \subsection[热机]{\itr{Heat Engine}{热机}}
            首先介绍一些基本概念
            \begin{Itemize}
                \item \itr{Cycle}{循环}是终态与始态相同的过程,其在$p-V$上的路径为一闭合曲线。
                \item \itr{Reversible Process}{可逆过程}是指存在逆过程使得系统与环境都恢复原状态的过程。
                可逆过程经过一系列无限接近平衡态的状态,且不存在热传导、气体的自由膨胀、扩散等过程。如果对一个过程,
                使系统恢复原状态一定会导致环境改变(能量耗散),则称为\itr{Irreversible Process}{不可逆过程}
                \item \itr{Heat Engine}{热机}是指各种利用内能做功的机械。
                \item \itr{Efficiency of Heat Engine}{热机效率}是一个循环内热机对外做功与其吸收的热的比值,即:
                \begin{equation}
                    e \overset{def}{=} \frac{W}{Q_{in}} = \frac{Q_{in}-Q_{out}}{Q_{in}}
                    \nonumber
                \end{equation}
                $Q_{in}$表示吸热,$Q_{out}$表示放热,注意这里取的都是绝对值。
                \item \itr{Perpetual Motion Machine}{永动机}是不需要物质与能量输入而可以永久向环境做功的热机,是无法实现的。
                第一类永动机试图寻找无能量消耗的热机,违反了能量守恒定律;第二类永动机则试图从单一热源取热并把它全部变为功,
                违反了热力学第二定律。
            \end{Itemize}
            然后我们来看两个经典的循环。

            \itr{Otto Cycle}{奥托循环}由两个等容过程与两个绝热过程组成,如图所示:
            \begin{singlefigure}[Otto Cycle]{Chapter7_otto.png}[0.4]
                AB与CD为绝热过程,BC与DA为等容过程,循环沿ABCDA方向进行
            \end{singlefigure}
            使用奥托循环的热机效率为:
            \begin{equation}
                e = 1 - \frac{T_{D} - T_{A}}{T_{C} - T_{B}}
                \nonumber
            \end{equation}
            
            \itr{Carnot Cycle}{卡诺循环}由两个等温过程与两个绝热过程组成,如图所示:
            \begin{singlefigure}[Carnot Cycle]{Chapter7_carnot.png}[0.4]
                AB与CD为等温过程,BC与DA为绝热过程,循环沿ABCDA方向进行时为正方向
            \end{singlefigure}
            使用正卡诺循环的热机称为\itr{Carnot Heat Engine}{卡诺热机},其热机效率为:
            \begin{equation}
                e = 1 - \frac{T_{c}}{T_{h}}
                \nonumber
            \end{equation}
            其中$T_c$为低温热源的温度,$T_h$为高温热源的温度。

            卡诺热机是所有热机中效率最高的,提高热机效率的手段为提高高温热源温度或降低低温热源温度。
            
            如果倒转卡诺循环,则得到卡诺冷机,此时其制冷系数为:
            \begin{equation}
                e^{\prime} \overset{def}{=} \frac{Q_c}{W} = \frac{T_c}{T_h - T_c}
                \nonumber
            \end{equation}
            同样的,其效率也是所有冷机中最高的。
        \subsection[熵]{\itr{Entropy}{熵}}
            \itr{Entropy}{熵}的变化量被定义为可逆过程中热与温度的比值,即:
            \begin{equation}
                \Delta S \overset{def}{=} \frac{Q_r}{T} 
                \nonumber
            \end{equation}
            或写作微分形式,即取近平衡态间转变的过程视为可逆过程:
            \begin{equation}
                \dif{S} = \frac{\dif{Q}}{T} 
                \nonumber
            \end{equation}
            
            熵是一个状态函数,熵的物理意义可以从下面这个公式中体现:
            \begin{equation}
                S = k_{_B}ln\Omega
                \nonumber
            \end{equation}
            其中$\Omega$表示所有可能的微观状态数量,也就是说,熵是系统可能的微观状态数量的度量,或者其他教材提到的“混乱度”。
            这个公式是连接微观与宏观间最重要的桥梁,也是统计力学的核心。

            考虑一个\itr{Free Expansion}{自由膨胀}过程,也就是说等温条件下气体向真空中膨胀。由于真空,气体自然无法做功,$W=0$;
            由于等温,$\Delta U=0$;那么由热力学第一定律,自然有$Q=0$。但是计算它的熵变,这个值为$\Delta S = nRln\frac{V_2}{V_1}>0$。
            众所周知,这个过程是不可逆的,也就是说那个$\Delta S<0$的逆过程是无法发生的,
            这正是\itr{Second Law of Thermodynamics}{热力学第二定律}表明的。
            \begin{law}[Second Law of Thermodynamics]
                孤立系统的熵不可能减少:
                \begin{equation}
                    \Delta S_{iso} \geq 0
                    \nonumber
                \end{equation}
                在可逆过程中取等号,熵保持不变;不可逆过程中熵一定增加。
            \end{law}
            这是热力学第二定律的本质,即熵增原理。此外热力学第二定律还有两种表述,所有表述之间相互等价。
            \begin{law}[Second Law of Thermodynamics——Clausius Statement]
                热不可能自发的从低温物体传到高温物体。

                或表述为:热不可能从低温物体传到高温物体,而不引起任何变化。
            \end{law}
            \begin{law}[Second Law of Thermodynamics——Kelvin Statement]
                不可能从单一热源取热,把它全部变为功而不产生其他任何影响。

                或表述为:第二类永动机不能实现。
            \end{law}
            熵给出了热的本质$\dif{Q} = T\dif{S}$,于是就有了下面的热力学基本公式:
            \begin{equation}
                \dif{U} = T\dif{S}-p\dif{V}
                \nonumber
            \end{equation}
            它联合了热力学的两大定律,是热力学最基本的关系之一。
    \section[热力学第三定律]{\itr{Third Law of Thermodynamics}{热力学第三定律}}
        我们再看一下卡诺循环的热机效率$e = 1-\frac{T_c}{T_h}$,注意到当低温热源$T_c = 0 \rm{K}$时,热机效率可以达到$100\%$。
        此时由于仍然是卡诺循环,我们并没有违反热力学第二定律,那么这是否意味着这是可行的呢?

        \itr{Third Law of Thermodynamics}{热力学第三定律}表明了这是无法实现的,因为我们无法找出一个\linebreak\itr{Absolute Zero}{绝对零度}
        的物体。
        \begin{law}[Third Law of Themiodynamics]
            绝对零度无法达到。

            或表述为:绝对零度下完美晶体的熵为$0$。
        \end{law}
        热力学第三定律意味着我们可以确定熵的零点,即将完美晶体在绝对零度下的熵定义为$0$,这样我们就可以确定熵的绝对值了。
    \section[信息]{\itr{Information}{信息}}
        本部分仅作简单介绍。
        
        根据我们的经验,信息量与事件发生概率成负相关,必然发生的事件比如\dove 的生日不在2月30日,就相当于什么都没说;
        信息量$Q$之间服从加法原则,事件概率$P$则服从乘法原则,故两者应为负对数关系:
        \begin{equation}
            Q = - logP
            \nonumber
        \end{equation}
        至于对数的底其实是任意的,彼此相差一个系数而已,通常选择的是$2$。

        至此我们引出\itr{Information Entropy$\backslash$ Shannon Entropy}{信息熵$\backslash$ 香农熵}的概念,
        它是事件发生前可能产生的信息量的数学期望,即:
        \begin{equation}
            S = E(Q) = \sum Q_{i}P_{i} = -\sum P_{i}logP_{i}
            \nonumber
        \end{equation}
        信息熵是信息的度量。

        然后再来了解一下\itr{Cross Entropy}{交叉熵},对于某事件,假设其有两种概率分布为$p(x)$与$q(x)$,
        则其交叉熵为:
        \begin{equation}
            H(p,q) = - \sum p({x_{i}})log(q({x_{i}}))
            \nonumber
        \end{equation}

        两个分布的差距为其中一个事件的概率分布与信息熵之差乘积的数学期望,即:
        \begin{equation}
            D_{KL}(p||q) = - \sum p({x_{i}})log(\frac{q({x_{i}})}{p({x_{i}})}) = H(p,p)-H(p,q)
            \nonumber
        \end{equation} 

        仅当$p$与$q$分布完全一致时差距为0,也就是说交叉熵刻画了两个分布的差距。在人工智能的逻辑回归中,
        其训练用的损失函数正是交叉熵损失函数。
%\end{document}
